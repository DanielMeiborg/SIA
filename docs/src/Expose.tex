\documentclass[12pt]{article}
\title{Exposé SIA Arbeit \\[1ex] \large ENTROPIE EINER WIRTSCHAFT}
\author{Daniel Meiborg}
\date{\today}

\usepackage{amsmath}
\usepackage[a4paper, top=2cm, left=2cm, right=2cm, bottom=2.5cm]{geometry}

\newcommand{\chapquote}[3]{\begin{quotation}
                               \textit{#1}
\end{quotation} \begin{flushright}
                    #2, \textit{#3}
\end{flushright} }


\begin{document}

    \maketitle
    \begin{quotation}
        \begin{flushright}
            \textit{You should call it entropy [...] no one really knows what
            entropy really is, so in a debate you will always have the advantage.}
        \end{flushright}
    \end{quotation}
    John von Neumann zu Claude Shannon,\textit{ Scientific American Vol. 225 No. 3,
        (1971)}

    \section*{Thema}

    \textit{Lassen sich einfache ökonomische Prozesse mit einem Markov-Prozess mit einer
    uniformen stationären Wahrscheinlichkeitsverteilung modellieren und aus der durch
    äußere Einflüsse entstehenden Entropiereduktion Rückschlüsse auf unsere Wirtschaft
    treffen?}


    \subsection*{Erläuterung}
    Grundbaustein dieser Herangehensweise ist der zweite Hauptsatz der Thermodynamik.
    Dieser gilt unter anderem für Markov-Prozesse (auch \textit{Markov Chains} genannt)
    unter bestimmten Voraussetzungen~\cite{cover1994processes}.
    Durch manuelles Eingreifen lässt sich die Entropie des Systems allerdings reduzieren.
    Diese Entropiereduktion ist äquivalent zu der Menge an Information, die man durch das
    Eingreifen erhält.
    Wenn man das in mehrere Subumgebungen unterteilt, kann man dadurch mehrere
    Wirtschaftstypen und ihre Eigenschaften vergleichen.

    \subsection*{Motivation}

    Ziel dieses Modells sind tiefere Erkenntnisse über das Grenzwertverhalten von
    Wirtschaften, sowie diese nach Typen basierend auf ihrer Entropie zu klassifizieren.

    \subsection*{Forschungsstand}

    Im bisherigen Forschungsstand wurden zwar schon Markov-Prozesse für die Modellierung
    von Wirtschaften verwendet, allerdings wurde dabei nicht auf die Entropie im oben
    beschriebenen Sinne geachtet~\cite{barde2020macroeconomic,Kostoska2020absorbingmc}.
    Genauso wurde auch das Entropieverhalten von Markov-Prozessen analysiert, aber nicht
    auf die Wirtschaft bezogen~\cite{Rahman2022mccharacteristics}.

    \newpage
    \section*{Zeitplan}
    \begin{description}
        \item[Recherche] Einlesen in das Themengebiet
        \item[Planung] Konzeptionierung des Modells und der Versuche
        \item[Framework] Programmierung des Frameworks für die Markov-Prozess-Analyse
        \item[Modellierung] Genaue Konfiguration/Eingabe der Parameter des Modells
        \item[Analyse] Untersuchung des Modells mit bisherigen Methoden
        \item[Manipulation] Eingreifen in die Simulation und Analyse der Entropie
        \item[Interpretation] Zurückführen der Ergebnisse auf die Wirtschaft
        \item[Wiederholung] Wiederholung mit anderen Modellen
    \end{description}

    \section*{Mögliche Probleme}
    \begin{description}
        \item[Modellierung] Die Markov-Eigenschaft ist nicht sinnvoll Modell erfüllbar.
        \item[Komplexität] Die benötigte Komplexitätsreduktion macht die Resultate
        unbrauchbar.
        \item[Speichereskalation] Durch zu viele Parameter wächst der Speicherbedarf zu
        hoch
    \end{description}
    \bibliography{main}
    \bibliographystyle{plain}

\end{document}
